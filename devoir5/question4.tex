\section*{5.1}
%\addcontentsline{toc}{section}{Question 1}

a) 
\begin{align*}
   \begin{pmatrix*}[r] 
   -1 & 0 \\
    1 & 1
   \end{pmatrix*}
   \cdot
   \begin{pmatrix*}[r] 
   1 \\
   0
   \end{pmatrix*}
   =
   \begin{pmatrix*}[r] 
   -1 \\
    1
   \end{pmatrix*}
\end{align*}

Le vecteur $v_1$ n'est pas un vecteur propre de la matrice, car aucun scalaire
$\lambda$ ne peut multiplier $v_1$ pour obtenir le résultat obtenu.

b) 
\begin{align*}
   \begin{pmatrix*}[r] 
   -1 & 0 \\
    1 & 1
   \end{pmatrix*}
   \cdot
   \begin{pmatrix*}[r] 
   0 \\
   1
   \end{pmatrix*}
   =
   \begin{pmatrix*}[r] 
   0 \\
   1 
   \end{pmatrix*}
\end{align*}

Le vecteur $v_2$ est un vecteur propre de la matrice avec $\lambda = 1$

c) 
\begin{align*}
   \begin{pmatrix*}[r] 
   -1 & 0 \\
    1 & 1
   \end{pmatrix*}
   \cdot
   \begin{pmatrix*}[r] 
   2 \\
   1
   \end{pmatrix*}
   =
   \begin{pmatrix*}[r] 
   -2 \\
    3 
   \end{pmatrix*}
\end{align*}

Le vecteur $v_3$ n'est pas un vecteur propre de la matrice, car aucun scalaire
$\lambda$ ne peut multiplier $v_3$ pour obtenir le résultat obtenu.

d) 
\begin{align*}
   \begin{pmatrix*}[r] 
   -1 & 0 \\
    1 & 1
   \end{pmatrix*}
   \cdot
   \begin{pmatrix*}[r] 
   -2 \\
   1
   \end{pmatrix*}
   =
   \begin{pmatrix*}[r] 
     2 \\
    -1 
   \end{pmatrix*}
\end{align*}

Le vecteur $v_4$ est un vecteur propre de la matrice avec $\lambda = -1$
