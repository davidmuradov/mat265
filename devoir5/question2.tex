\section*{4.14}
%\addcontentsline{toc}{section}{Question 1}

f) L'ÉDO à résoudre est :
\begin{align*}
    y''' - 13y' - 12y = 2e^{-3x} - 5e^{-4x}
\end{align*}

L'équation caractéristique est :
\begin{align*}
    m^3 - 13m -12 = 0
\end{align*}

Les racines sont :
\begin{align*}
    (m+1)(m-4)(m+3) &= 0 \\
    m_{1,2,3} &= -1,-3,4
\end{align*}

La solution homogène est :
\begin{align*}
    y_h = C_1e^{-x}+C_2e^{-3x}+C_3e^{4x}
\end{align*}

Le candidat de la solution particulière choisit est :
\begin{align*}
    y_p = 2Axe^{-3x} - 5Be^{-4x}
\end{align*}

En passant par dessus toutes les étapes de dérivations et d'algèbre, on trouve
l'équation suivante en insérant $y_p$ dans l'ÉDO :
\begin{align*}
    28Ae^{-3x} +120Be^{-4x} &= 2e^{-3x} +-5e^{-4x} \\
    A = \f{1}{14}&,\ B = -\f{1}{24}
\end{align*}

D'où :
\begin{align*}
    y_p = \f{1}{7}xe^{-3x} + \f{5}{24}e^{-4x}
\end{align*}

La solution finale est :
\begin{align*}
    y = C_1 e^{-x} + C_2 e^{-3x} + C_3 e^{4x} +\f{1}{7}xe^{-3x} + \f{5}{24}e^{-4x}
\end{align*}

g) L'ÉDO à résoudre est :
\begin{align*}
    2x'' +4x' = 3t + 4e^{-t}
\end{align*}

Son équation caractéristique est :
\begin{align*}
    2m^2 + 4m = 0
\end{align*}

Les racines sont :
\begin{align*}
    m(m+2) &= 0 \\
    m_{1,2} &= 0,-2
\end{align*}

La solution homogène est :
\begin{align*}
    x_h = C_1 + C_2 e^{-2t}
\end{align*}

Étant donné qu'il n'y a que des dérivées dans l'ÉDO pour $x$, on choisit le
candidat modifié suivant :
\begin{align*}
    x_p = At^2 + Bt + C + De^{-t}
\end{align*}

En calculant les dérivées sur $x_p$ et en insérant dans l'ÉDO, on trouve :
\begin{align*}
    4A + 2De^{-t} + 8At + 4B - 4De^{-t} = 3t + 4e^{-t}
\end{align*}

D'ici il en découle que :
\begin{align*}
    D=-2,\ A=\f{3}{8},\ B=-\f{3}{8},\ C = \alpha,\ \alpha\in \mathbb{R}
\end{align*}

La solution générale proposée est :
\begin{align*}
    x = C_1 + C_2 e^{-2t} +\f{3}{8}t^2 - \f{3}{8}t - 2e^{-t}
\end{align*}

En utilisant les conditions initiales données, on trouve que :
\begin{align*}
    0 &= C_1 + C_2 - 2 \\
    -\f{1}{4} &= -2C_2 -\f{3}{8} + 2 \\
    \implies& C_2 = \f{15}{16},\ C_1 = \f{17}{16}
\end{align*}

La solution devient :
\begin{align*}
    x = \f{17}{16} +\f{15}{16}e^{-2t} +\f{3}{8}t^2 -\f{3}{8}t -2e^{-t}
\end{align*}
