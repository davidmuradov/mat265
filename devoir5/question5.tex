\section*{5.2}
%\addcontentsline{toc}{section}{Question 1}

a) Soit la matrice :
\begin{align*}
    A =
    \begin{pmatrix*}[r]
	1&1\\
	4&-2
    \end{pmatrix*}
\end{align*}

On calcule les valeurs propres avec la Ti :
\begin{align*}
    \linalgcas\eigenvals(A)\implies\lambda_{1,2} = -3,2
\end{align*}

Le vecteur propre associé à $\lambda = -3$ est :
\begin{align*}
    v_1 = \begin{pmatrix*}[r]
    	1\\-4
    \end{pmatrix*}
\end{align*}

Le vecteur propre associé à $\lambda = 2$ est :
\begin{align*}
    v_2 = \begin{pmatrix*}[r]
    	1\\1
    \end{pmatrix*}
\end{align*}

Puisque les deux valeurs propres sont réelles et distinctes, 
la solution au système d'équation est :
\begin{align*}
    \begin{pmatrix*}[r]
        x_1\\x_2
    \end{pmatrix*}
    =
    C_1
    \begin{pmatrix*}[r]
        1\\-4
    \end{pmatrix*}
    e^{-3t}
    +C_2
    \begin{pmatrix*}[r]
        1\\1
    \end{pmatrix*}
    e^{2t}
\end{align*}

b) Soit la matrice :
\begin{align*}
    A =
    \begin{pmatrix*}[r]
	-1&-4\\
	1&-1
    \end{pmatrix*}
\end{align*}

On calcule les valeurs propres avec la Ti :
\begin{align*}
    \linalgcas\eigenvals(A)\implies\lambda_{1,2} = -1+2i,-1-2i
\end{align*}

Le vecteur propre associé à $\lambda = -1+2i$ est :
\begin{align*}
    v_1 = \begin{pmatrix*}[r]
    	1\\-\f{i}{2}
    \end{pmatrix*}
\end{align*}

Le vecteur propre associé à $\lambda = -1-2i$ est :
\begin{align*}
    v_2 = \begin{pmatrix*}[r]
    	1\\\f{i}{2}
    \end{pmatrix*}
\end{align*}

On pose aussi :
\begin{align*}
    v_{\lambda_1} = u_1 + iu_2 =
    \begin{pmatrix*}[r]
        1\\0
    \end{pmatrix*}
    +i
    \begin{pmatrix*}[r]
        0\\-\f{1}{2}
    \end{pmatrix*}
\end{align*}

Puisque les deux valeurs propres sont complexes et distinctes, 
la solution au système d'équation est :
\begin{align*}
    \begin{pmatrix*}[r]
        x_1\\x_2
    \end{pmatrix*}
    =
    C_1 e^{-t}\prt{
	\begin{pmatrix*}[r]
	    1\\0
	\end{pmatrix*}\cos(2t)
	+
	\begin{pmatrix*}[r]
	    0\\\f{1}{2}
	\end{pmatrix*}\sin(2t)
    }
    +C_2 e^{-t}\prt{
	\begin{pmatrix*}[r]
	    1\\0
	\end{pmatrix*}\sin(2t)
	+
	\begin{pmatrix*}[r]
	    0\\-\f{1}{2}
	\end{pmatrix*}\cos(2t)
    }
\end{align*}

c) Soit la matrice :
\begin{align*}
    A =
    \begin{pmatrix*}[r]
	3&-4\\
	1&-1
    \end{pmatrix*}
\end{align*}

On calcule les valeurs propres avec la Ti :
\begin{align*}
    \linalgcas\eigenvals(A)\implies\lambda = 1
\end{align*}

Le vecteur propre associé à $\lambda = 1$ est :
\begin{align*}
    v_1 = \begin{pmatrix*}[r]
    	1\\\f{1}{2}
    \end{pmatrix*}
\end{align*}

On calcule $(A-\lambda I)$ :
\begin{align*}
    \begin{pmatrix*}[r]
	3&-4\\
	1&-1
    \end{pmatrix*} -
    \begin{pmatrix*}[r]
	1&0\\
	0&1
    \end{pmatrix*} =
    \begin{pmatrix*}[r]
	2&-4\\
	1&-2
    \end{pmatrix*}
\end{align*}

Soit $w\in\mathbb{R}^2$. On résout le système d'équation suivant :
\begin{align*}
    2w_1 - 4w_2 &= 1 \\
    w_1 - 2w_2 &= \f{1}{2}
\end{align*}

Un vecteur $w$ possible qui résout ce système est :
\begin{align*}
    w = (5/2, 1)
\end{align*}

Puisque la matrice n'admet qu'une seule valeur propre,
la solution au système d'équation est :
\begin{align*}
    \begin{pmatrix*}[r]
        x_1\\x_2
    \end{pmatrix*} = 
    C_1
    \begin{pmatrix*}[r]
        1\\\f{1}{2}
    \end{pmatrix*}e^{t}
    +C_2\prt{
	t
	\begin{pmatrix*}[r]
	    1\\\f{1}{2}
	\end{pmatrix*}
	+
	\begin{pmatrix*}[r]
	    \f{5}{2}\\1
	\end{pmatrix*}
    }e^{t}
\end{align*}
