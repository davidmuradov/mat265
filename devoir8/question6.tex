\section*{10.1}
%\addcontentsline{toc}{section}{Question 1}

a) On tente de résoudre l'équation suivante :
\begin{align*}
    u_x -2u_y = 0,\ u(0,y) = 2y
\end{align*}

On pose :
\begin{align*}
    a(x,y) = 1,\ b(x,y) = -2
\end{align*}

On cherche à trouver la solution de l'équation suivante :
\begin{align*}
    \dv{y}{x} = \f{b(x,y)}{a(x,y)} = -2 \implies y = -2x +C \iff y + 2x = C
\end{align*}

On pose :
\begin{align*}
    g(x,y) = y+2x
\end{align*}

d'où :
\begin{gather*}
    u(x,y) = f(g(x,y)) = f(y+2x)\\
    u(0,y) = 2y = f(y + 0) = f(y)\\
    u(x,y) = 2(y+2x) = 4x+2y
\end{gather*}

\section*{10.2}
%\addcontentsline{toc}{section}{Question 1}

Pour une fonction $F(t)$ de forme sinusoïdale, le régime permanent est :
\begin{align*}
    y_p = \f{F_0/m}{\sqrt{(\omega^2 -\omega_0^2)^2 +\f{b\omega^2}{m}}}
    \cos\prt{\omega t + \delta}
\end{align*}

Soit la nouvelle fonction $G(t) = 2F(t)$, alors :
\begin{align*}
    G(t) = 2F_0 \cos\prt{\omega t}
\end{align*}

et son régime permanent est :
\begin{align*}
    y_{pg} = \f{2F_0/m}{\sqrt{(\omega^2 -\omega_0^2)^2 +\f{b\omega^2}{m}}}
    \cos\prt{\omega t + \delta} = 2y_p
\end{align*}

On remarque que l'amplitude est deux fois plus grande. Cependant, ceci n'est
vérifié que pour une excitation sinusoïdale, il faudrait passer par un calcul
plus détaillé pour observer ce qui se produit pour une fonction $F(t)$
quelconque (par exemple, en calculant la transformée de Laplace de l'ÉDO et
en vérifiant si un régime transitoire et permanent existent et refaire le
calcule pour $G(t) = 2F(t)$ pour produire des observations).
