\section*{2.13}
%\addcontentsline{toc}{section}{Question 1}

c) On tente de résoudre :
\begin{align*}
    y' -2y &= 4\cos(3x)
\end{align*}

C'est une équation linéaire d'ordre 1 qu'on peut résoudre avec un facteur
intégrant. On pose les paramètres suivants :
\begin{align*}
    P(x) = -2,\quad Q(x) = 4\cos(3x),\quad \mu(x) = e^{\int{P(x)dx}} =
    e^{-2x}
\end{align*}

La solution générale est :
\begin{align*}
    \varphi(x) = Ce^{2x} + \dfrac{4}{e^{-2x}}\int{e^{-2x}\cos(3x)\ dx}
\end{align*}

La dernière intégrale se résout avec une intégration par partie. On utilise la
méthode tabulaire :
\begin{table}[H]
    \centering
    %\caption{Mesures de tensions et de courants ainsi que le calcul du gain selon le courant $I_{\text{B}}$}
    \begin{tabular}{rrr}
	\toprule[1pt]
	 & D & I \\
	\midrule
	+ & $\cos(3x)$ & $e^{-2x}$\vspace{1mm}\\
	- & $-3\sin(3x)$ & $-\dfrac{1}{2}e^{-2x}$\vspace{2mm}\\
	+ & $-9\cos(3x)$ & $\dfrac{1}{4}e^{-2x}$\vspace{2mm}\\
	\bottomrule[1pt]
    \end{tabular}
\end{table}

D'ici on déduit que :
\begin{align*}
    \int{e^{-2x}\cos(3x)\ dx} &= -\dfrac{1}{2}\cos(3x)e^{-2x} 
    +\dfrac{3}{4}\sin(3x)e^{-2x} -\dfrac{9}{4}\int{e^{-2x}\cos(3x)\ dx} \\
    \int{e^{-2x}\cos(3x)\ dx} &= -\dfrac{2}{13}\cos(3x)e^{-2x} 
    +\dfrac{3}{13}\sin(3x)e^{-2x}
\end{align*}

En insérant cette égalité dans la solution générale, on trouve l'expression
finale :
\begin{align*}
    \varphi(x) = Ce^{2x} - \dfrac{8}{13}\cos(3x) + \dfrac{12}{13}\sin(3x)
\end{align*}
