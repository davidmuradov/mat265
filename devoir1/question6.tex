\section*{1.2}
%\addcontentsline{toc}{section}{Question 1}

Soit $r$ la longueur d'une arête. L'aire d'une des faces est $r^2$. Étant donné
que la longueur des arêtes décroit à un taux égal à l'aire d'une face, on déduit
l'équation différientielle suivante :
\begin{align*}
    \dfrac{dr}{dt} &= -r^2
\end{align*}

C'est une équation à variables séparables. On la résout avec la condition
initiale $r(0)=1$ :
\begin{align*}
    \dfrac{dr}{dt} &= -r^2 \\
    -\int{r^{-2}\ dr} &= \int{1\ dt} \\
    r^{-1} &= t + C \\
    r &= \dfrac{1}{t+C}
\end{align*}

On calcule la valeur de $C$ :
\begin{align*}
    r(0) = 1 &= \dfrac{1}{1+C} \\
    C &= 1 \\
    \implies r(t) &= \dfrac{1}{t+1}
\end{align*}

Le volume du cube $V(t)$ pour tout $t$ est alors :
\begin{align*}
    V(t) = \dfrac{1}{(t+1)^3}
\end{align*}
