\section*{1.1}
%\addcontentsline{toc}{section}{Question 1}

a) On tente de résoudre :
\begin{align*}
    y' + y &= x
\end{align*}

C'est une équation linéaire d'ordre 1 qu'on peut résoudre avec un facteur
intégrant. On pose les paramètres suivants :
\begin{align*}
    P(x) = 1,\quad Q(x) = x,\quad \mu(x) = e^{\int{P(x)dx}} =
    e^{x}
\end{align*}

La solution générale est :
\begin{align*}
    \varphi(x) &= Ce^{-x} + e^{-x}\int{xe^{x}\ dx} \\
    \varphi(x) &= Ce^{-x} + e^{-x}(xe^{x} - e^x) \\
    \varphi(x) &= Ce^{-x} + x - 1
\end{align*}

On cherche à trouver la solution particulière avec $\varphi(0) = 1$ :
\begin{align*}
    \varphi(0) = 1 &= Ce^{0} + 0 - 1 \\
    1 &= C - 1 \\
    C &= 2
\end{align*}

La solution particulière est :
\begin{align*}
    \varphi(x) &= 2e^{-x} + x - 1 \\
\end{align*}


b) On tente de résoudre :
\begin{align*}
    y' &= \dfrac{1}{2xy +2y -x -1}
\end{align*}

C'est une équation à variable séparable. Pour le voir, on procède avec quelques
factorisations :
\begin{align*}
    y' &= \dfrac{1}{2xy +2y -x -1} \\
    &= \dfrac{1}{2y(x + 1) -(x +1)} \\
    &= \dfrac{1}{(x+1)(2y-1)}
\end{align*}

L'équation à résoudre est donc :
\begin{align*}
    \int{(2y-1)\ dy} &= \int{\dfrac{1}{x+1}\ dx} \\
    y^2 -y &= \ln(|x+1|) + C \\
    y(y-1) &= \ln(|x+1|) + C
\end{align*}

En posant $y(0)=1$, on trouve la solution particulière :
\begin{align*}
    1(1-1) &= \ln(1) + C \\
    C &= 0 \\
    \implies 
    y(y-1) &= \ln(|x+1|)
\end{align*}
