\section*{2.8}
%\addcontentsline{toc}{section}{Question 1}

a) La forme générale du taux de variation de la quantité d'acide dans un
contenant est :
\begin{align*}
    \dv{q}{t} = \text{taux en entrée}-\text{taux en sortie}
\end{align*}

On pose les conditions suivantes selon le contexte :
\begin{align*}
    C_e &= 0.15,\ D_e = 5 \\
    D_s &= 6 \\
    q(t_0) &= 0.004\cdot 100 = 0.4
\end{align*}

Le taux d'acide en entrée est :
\begin{align*}
    C_e\cdot D_e = 5\cdot0.15 = 0.75
\end{align*}

Le taux d'acide en sortie dépend de la quantité actuelle d'acide et du volume
total. En particulier, la quantité d'acide en tout temps est $q$ alors que le
volume total diminue de 1 litre à chaque minute. Le taux de sortie est donc :
\begin{align*}
    6\cdot\f{q}{100-t}
\end{align*}

L'équation différentielle devient :
\begin{align*}
    \dv{q}{t} = 0.75 - 6\cdot\f{q}{100-t}
\end{align*}

On résout avec la Ti :
\begin{align*}
    \desolve\prt{\dv{q}{t} = 0.75 - 6\cdot\f{q}{100-t}\text{ and }
    q(0) = 0.4,t,q}
\end{align*}

On obtient :
\begin{align*}
    q(t) = 0.4 + 0.726t - 0.0219t^2 + 2.92\cdot10^{-4}t^3
    -2.19\cdot10^{-6}t^4 + 8.76\cdot10^{-9}t^5
    -1.46\cdot10^{-11}t^6
\end{align*}

b) La concentration atteint 10\% quand :
\begin{align*}
    \f{q(t)}{100-t} = 0.1
\end{align*}

On fait un solve avec la Ti :
\begin{align*}
    \solve\prt{\f{q(t)}{100-t} = 0.1,t}
\end{align*}

On trouve $t=19.29$ minutes.
