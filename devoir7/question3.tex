\section*{5.15}
%\addcontentsline{toc}{section}{Question 1}

b) On tente de calculer l'expression suivante :
\begin{align*}
    \lap^{-1}\prt{\f{e^{-5s}}{s^3} + \prt{\f{5s-1}{s^2+4}}e^{-4s}} = \lap^{-1}{\prt{f(t)}}
\end{align*}

On procède à quelques manipulations algébriques :
\begin{align*}
    \lap^{-1}{\prt{f(t)}}
    &= \lap^{-1}\prt{\f{2e^{-5s}}{2s^3} + \prt{\f{5s-1}{s^2+4}}e^{-4s}} \\
    &= \f{1}{2}\lap^{-1}\prt{e^{-5s}\f{2}{s^{2+1}}}
    + \lap^{-1}\prt{\f{5s-1}{s^2+4}\cdot e^{-4s}} \\
    &= \f{1}{2}(t-5)^2 u(t-5)
    + \lap^{-1}\prt{e^{-4s}\prt{5\cdot\f{s}{s^2 + 2^2} -\f{1}{s^2 + 2^2}}}
\end{align*}

La transformée inverse de :
\begin{align*}
    5\cdot\f{s}{s^2 + 2^2} -\f{1}{s^2 + 2^2}
\end{align*}

nous donne la fonction inverse :
\begin{align*}
    F(s) = 5\cos(2t) - \f{1}{2}\sin(2t)
\end{align*}

d'où :
\begin{align*}
    \lap^{-1}{\prt{f(t)}} &=
    \f{1}{2}(t-5)^2 u(t-5)
    + \lap^{-1}\prt{e^{-4s}F(s)} \\
    &= \f{1}{2}(t-5)^2 u(t-5) +
    \prt{5\cos(2(t-4)) - \f{1}{2}\sin(2(t-4))}u(t-4)
\end{align*}
