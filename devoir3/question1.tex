\section*{2.8}
%\addcontentsline{toc}{section}{Question 1}

Ceci est un problème de refroidissement de Newton. La température ambiante fixe
est celle de l'air dehors qui est à $-15$. Le millieu qui se refroidit est la
température à l'intérieur de la chambre. On pose les conditions initiales
suivantes :
\begin{align*}
    T(0) = 20,\quad T(1) = 16,\quad T_A = -15
\end{align*}

La solution à l'équation différientielle qui régit ce phénomène est :
\begin{align*}
    T(t) = T_A +Ce^{kt}
\end{align*}

On calcule la valeur de $C$ :
\begin{gather*}
    T(0) = 20 = -15 +Ce^{0} \\
    C = 35
\end{gather*}

On calcule la valeur de $k$ :
\begin{align*}
    T(1) = 16 &= -15 +35e^{k} \\
    35e^k &= 31 \\
    e^k &= \f{31}{35} \\
    k &= \ln\prt{\f{31}{35}} = -0.121
\end{align*}

La fonction de température est donc :
\begin{align*}
    T(t) = -15 + 35e^{-0.121t}
\end{align*}

À 5h le matin, la température est :
\begin{align*}
    T(6) = -15 + 35e^{-0.121\cdot 6} = 1.898
\end{align*}
