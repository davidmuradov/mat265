\section*{3.5}
%\addcontentsline{toc}{section}{Question 1}

a) On pose les conditions initiales suivantes :
\begin{align*}
    g = 9.81,\ m = 90,\ v(0) = 50,\ x(0) = 0
\end{align*}

Le référentiel utilisé est celui où la position initiale du parachutiste est
de 0 et pour lequel l'axe de mouvement rectiligne est orienté vers le centre
de la terre (tout phénomène orienté vers le sol est considéré positif). Les deux
forces qui agissent sur le parachutiste sont :
\begin{align*}
    F_1 = mg,\ F_2 = -20v^2
\end{align*}

La loi de Newton donne :
\begin{align*}
    \sum{F} &= mg - 20v^2 = m\dv{v}{t} \\
    \iff v' &= g - \f{20}{m}v^2
\end{align*}

On utilise la TI pour trouver la solution de cette EDO. On tape la commande
suivante pour obtenir directement $v(t)$ :
\begin{align*}
    \solve\prt{\desolve\prt{v'=g-\f{20}{m}\cdot v^2\ \text{and }
    v(0)=50, t, v},v}
\end{align*}

On obtient alors deux solutions :
\begin{align*}
    v(t) = 6.644\cdot\f{19.163^t - 0.765}{19.163^t + 0.765}\ \text{ou }
    v(t) = 6.644\cdot\f{19.163^t + 0.765}{19.163^t - 0.765}
\end{align*}

En posant $t=0$, on trouve que la solution qui donne la bonne vitesse initiale
de 50 est :
\begin{align*}
    v(t) = 6.644\cdot\f{19.163^t + 0.765}{19.163^t - 0.765}
\end{align*}

b) Pour trouver la vitesse terminale, il suffit de prendre la limite quand $t$
tend vers l'infini :
\begin{align*}
    \lim_{t\to\infty} v(t) &= \lim_{t\to\infty}
    6.644\cdot\f{19.163^t + 0.765}{19.163^t - 0.765} \\
    &= 6.644\cdot\lim_{t\to\infty}\f{19.163^t + 0.765}{19.163^t - 0.765} \\
    &= 6.644\cdot\lim_{t\to\infty}\f{1 + \f{0.765}{19.163^t}}{1 - \f{0.765}{19.163^t}} \\
    &= 6.644\cdot 1^+ \\
    v_l&= 6.644
\end{align*}

c) Pour trouver le temps pour atterrir au sol, il faut résoudre l'EDO suivante :
\begin{align*}
    x'(t) = 6.644\cdot\f{19.163^t + 0.765}{19.163^t - 0.765}
\end{align*}

On utilise la TI pour la résoudre :
\begin{align*}
    \solve\prt{\desolve\prt{x'=6.644\cdot\f{19.163^t + 0.765}{19.163^t - 0.765}
    \ \text{and } x(0)=0,t,x},x}
\end{align*}

Ensuite, il suffit de faire un solve pour $t$ sur le résultat obtenu pour $x=1500$.
Puisque la calculatrice peut avoir des difficultés à trouver une solution, on lui
donne une contrainte sur $t$. La contrainte a été trouvée par essais erreurs :
\begin{align*}
    \solve\prt{1500 = 4.5\cdot\ln(((4.378^t-0.874\cdots))),t}|\ t>200
\end{align*}

On obtient $t = 224.79$ secondes.
