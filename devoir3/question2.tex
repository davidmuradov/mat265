\section*{2.37}
%\addcontentsline{toc}{section}{Question 1}

e) L'EDO initiale est :
\begin{align*}
    \sqrt{1 + x^3}\ \dv{y}{x} = x^2 y + x^2
\end{align*}

En factorisant par $x^2$, on voit que l'équation est à variables séparables :
\begin{align*}
    \sqrt{1 + x^3}\ \dv{y}{x} &= x^2(y + 1) \\
    \f{1}{y+1}\dv{y}{x} &= \f{x^2}{\sqrt{1 + x^3}}
\end{align*}

L'équation est aussi une équation linéaire :
\begin{align*}
    \sqrt{1 + x^3}\ \dv{y}{x} -x^2 y &= x^2 \\
    \dv{y}{x} -\f{x^2}{\sqrt{1 + x^3}}y &= \f{x^2}{\sqrt{1 + x^3}}
\end{align*}

On résout l'équation à variables séparables :
\begin{align*}
    \int{\f{1}{y+1}\ dy} &= \int{\f{x^2}{\sqrt{1 + x^3}}\ dx} \\
    \ln(y+1) &= \int{\f{x^2}{\sqrt{1 + x^3}}\ dx} +C \\
\end{align*}

On pose :
\begin{align*}
    u = 1+x^3,\quad du = 3x^2\ dx,\quad x^2\ dx = \f{1}{3}\ du
\end{align*}

L'intégrale devient :
\begin{align*}
    \int{\f{x^2}{\sqrt{1 + x^3}}\ dx} &= \f{1}{3}\int{\f{1}{\sqrt{u}}\ du} =
    \f{2}{3}\sqrt{u} = \f{2}{3}\sqrt{1+x^3}\\
\end{align*}

Donc :
\begin{align*}
    \ln(y+1) &= \f{2}{3}\sqrt{1+x^3} +C \\
    y+1 &= C_1 e^{\frac{2}{3}\sqrt{1+x^3}} \\
    y &= C_1 e^{\frac{2}{3}\sqrt{1+x^3}} - 1
\end{align*}

e) L'EDO initiale est :
\begin{align*}
    y' = x(x+y)
\end{align*}

C'est une équation linéaire d'ordre 1 :
\begin{align*}
    y' &= x^2+xy \\
    y' -xy &= x^2
\end{align*}

On pose :
\begin{align*}
    P(x) = -x,\ Q(x) = x^2
\end{align*}

Le facteur intégrant est :
\begin{align*}
    \mu(x) = e^{\int{P(x)\ dx}} = e^{\int{-x\ dx}} = e^{-\frac{1}{2}x^2}
\end{align*}

La solution à l'équation différientielle est :
\begin{align*}
    y = Ce^{\frac{1}{2}x^2} + e^{\frac{1}{2}x^2}\int{x^2 e^{-\frac{1}{2}x^2}\ dx}
\end{align*}

Ceci est la solution donnée dans le manuel. L'intégrale non résolue ne s'exprime
pas en termes de fonctions élémentaires, mais on peut quand même essayer de
trouver une solution non triviale. On tente de résoudre l'intégrale :
\begin{align*}
    \int{x^2 e^{-\frac{1}{2}x^2}\ dx} = \int{x\cdot xe^{-\frac{1}{2}x^2}\ dx}
\end{align*}

On fait l'intégration par parties:
\begin{table}[H]
    \centering
    %\caption{Mesures de tensions et de courants ainsi que le calcul du gain selon le courant $I_{\text{B}}$}
    \begin{tabular}{rrr}
	\toprule[1pt]
	 & D & I \\
	\midrule
	+ & $x$ & $xe^{-\frac{1}{2}x^2}$\vspace{1mm}\\
	- & $1$ & $-e^{-\frac{1}{2}x^2}$\vspace{2mm}\\
	+ & $0$ & \vspace{2mm}\\
	\bottomrule[1pt]
    \end{tabular}
\end{table}

Pour compléter le tableau, il faut résoudre l'intégrale suivante :
\begin{align*}
    -\int{e^{-\frac{1}{2}x^2}\ dx}
\end{align*}

Étant donné le $x^2$ dans l'exponentielle, on s'attend à obtenir la fonction
d'erreur de Gauss. On voudrait trouver une valeur $u$ telle que :
\begin{align*}
    u^2 &= \f{1}{2}x^2 \\
    u &= \f{x}{\sqrt{2}}
\end{align*}

On utilise ce $u$ pour faire une substitution :
\begin{align*}
    du &= \f{dx}{\sqrt{2}},\ dx = \sqrt{2}\ du
\end{align*}

L'intégrale devient :
\begin{align*}
    -\int{e^{-\frac{1}{2}x^2}\ dx} &= -\sqrt{2}\int{e^{-u^2}\ du} \\
    &= -\sqrt{2}\cdot\f{\sqrt{\pi}}{2}\cdot\f{2}{\sqrt{\pi}}\int{e^{-u^2}\ du} \\
    &= -\sqrt{2}\cdot\f{\sqrt{\pi}}{2}\cdot\erf(u) \\
    &= -\f{\sqrt{\pi}}{\sqrt{2}}\cdot\erf(u) \\
    &= -\sqrt{\f{\pi}{2}}\erf(u) \\
    &= -\sqrt{\f{\pi}{2}}\erf\prt{\f{x}{\sqrt{2}}}
\end{align*}

On complète le tableau d'intégration par parties :
\begin{table}[H]
    \centering
    %\caption{Mesures de tensions et de courants ainsi que le calcul du gain selon le courant $I_{\text{B}}$}
    \begin{tabular}{rrr}
	\toprule[1pt]
	 & D & I \\
	\midrule
	+ & $x$ & $xe^{-\frac{1}{2}x^2}$\vspace{1mm}\\
	- & $1$ & $-e^{-\frac{1}{2}x^2}$\vspace{2mm}\\
	+ & $0$ & $-\sqrt{\f{\pi}{2}}\erf\prt{\f{x}{\sqrt{2}}}$\vspace{2mm}\\
	\bottomrule[1pt]
    \end{tabular}
\end{table}

La solution finale est donc :
\begin{align*}
    y(x) &= Ce^{\frac{1}{2}x^2} + e^{\frac{1}{2}x^2}
    \prt{-xe^{-\frac{1}{2}x^2} + \sqrt{\f{\pi}{2}}\erf\prt{\f{x}{\sqrt{2}}}} \\
    &= Ce^{\frac{1}{2}x^2} 
    -x + \sqrt{\f{\pi}{2}}\erf\prt{\f{x}{\sqrt{2}}}e^{\frac{1}{2}x^2} \\
\end{align*}
